\documentclass[titlepage]{article}
\usepackage{graphicx}
\usepackage{amsmath}
\usepackage{biblatex}

\begin{document}

\title{Experiment 1: Mini Report}
\author{Tian Ye \\ \\ UID: 704931660 \\ \\ TA: Wen Li Wen \\ \\ Lab 8 Tuesday 6:00 PM}
\date{October 10th, 2017}

\maketitle

\section*{Introduction}

While it is not completely evident upon first glance, modern day accelerometers are ubiquitous. From being found in motion-sensing games, HDD protection inside laptops, airbag deployment,$^1$ to more niche fields such as avionics,$^2$ animal studies,$^3$ and health monitoring,$\textsuperscript{4,5}$ the ubiquity of accelerometers in many of today's technologies serves as a primary driving point for its development. 

\subsection*{Mechanics of Accelerometers}

While modern accelerometers have evolved greatly in comparison to their older cousins thanks to the advancement in the microelectromechanical systems (MEMS),$\textsuperscript{2,4}$ the core premise of the modern accelerometer remains the same. 
Accelerometers operate by measuring linear acceleration of an object along its axes, typically the dimensions of x, y, and z.$^1$ 

One type of the modern day accelerometer, the piezoresistive type, operates via the measurement of elongation or compression of a series of stress beams connected to a mass. Pizoresistors are placed at the end of the stress beams, where the maximum stress regions are located. As the mass moves, the beams compress, causing the resistance of the pizoresistors to change.$^2$
From this point it is a matter of conversion of a voltage output to acceleration, a task similar to what was undertaken in Experiment 0.

While the text above only refers to a single type of modern accelerometers, the common trend of miniaturization and integration in technology can be found across nearly all types of accelerometers as well.
Thanks to the advancement of modern material sciences, accelerometers have become relatively easy to fabricate and extremely miniaturized, even to the point of being embedded into fabric as an e-textile system.$\textsuperscript{2,4}$ Advances in the field of material science also indicate the future possibility of printing a full circuit board on fabric.$^1$

Another feature of accelerometers is that different accelerometers measure acceleration with different precision corresponding to the application of each accelerometer. In avionics, accelerometers are able to measure change in acceleration of $\pm$ 10g, with a maximum acceleration being withstood being  $\pm$ 50g,$^2$ while in other fields regarding acceleration of a much smaller magnitude, such as that of an animal, the accuracy being of the magnitude of $\pm$ 0.06g.$^3$ The versatility of the precision of accelerometers allows it to be employed in a variety of fields for many different purposes - a key reason for its pervasiveness.

With increased miniaturization of accelerometers, the application of it, while still being very much relevant in its traditional fields of avionics and automobiles, has begun shifting towards the field of monitoring as well.
The field that accelerometer usage is currently shifting towards is in the field of health monitoring for both children$^5$ and those with health disabilities and/or need of rehabilitation.$^4$ 

\pagebreak

\subsection*{Application}

One of the primary obstacles facing American society today in regards to children is the high obesity rate, and accelerometers of the triaxial nature are well suited to measuring activity that does not necessarily contain movement in the vertical plane, as do traditional accelerometers. This allows the measurement of physical activity besides those that involve movement in the vertical plane, such as cycling and swimming.$^5$

In the field of healthcare, accelerometers are able to measure a particular patient's health data and relay the information to hospitals, families, and clinicians via the usage of accelerometers integrated into clothing. This in turn allows caregivers to implement intervention measures as need be, and can potentially in the future lead to remote monitoring of patients' health.$^4$ 

From these applications it is possible to extrapolate future applications of accelerometers as they become more and more miniaturized and sensitive, capable of being integrated into everyday clothing and measuring the slightest of change in body systems and reporting it to emergency personnel, if need be.$^4$

\section*{Conclusion}

The application of accelerometers, however, need not be isolated to only their current fields in the future. The fact that it is nearly universal now only serves as a statement to their future ubiquity, and consequently, the importance of their development for the future.

\section*{}
648

\pagebreak

\begin{thebibliography}{1}
\bibitem{a}
Gerhard, D. Three Degrees of “G”s: How an Airbag Deployment Sensor Transformed Video Games, Exercise, and Dance. \textit{M/C Journal}, 16:6, 2013. 
\bibitem{b}
Sharma, A., Mukhiya, R., Kumar, S. S., Pant, B. D. Design and Simulation of Bulk Micromachined Accelerometer for Avionics Application. Arya College of Engineering and IT, CSIR-Central Electronics Engineering Research Institute (CEERI). Available online at https://link.springer.com/content/pdf/10.1007$\%$2F978-3-642-42024-5$\_$12.pdf (2013).
\bibitem{c}
Shepard, E. L. C. \textit{et al}. Identification of Animal Movement Patterns Using Tri-Axial Accelerometry. Preprint at http://www.int-res.com/articles/esr2008/theme/Tracking/TMVpp1.pdf (2008).
\bibitem{d}
Patel, S., Park, H., Bonato, P., Chan, L., Rodgers, M. A Review of Wearable Sensors and Systems with Application in Rehabilitation.\textit{Journal of NeuroEngineering and Rehabilitation} https://doi.org/10.1186/1743-0003-9-21 (2011).
\bibitem{e}
Robertson, W. \textit{et al}. Utility of Accelerometers to Measure Physical Activity in Children Attending an Obesity Treatment Intervention. \textit{Journal of Obesity} http://dx.doi.org/10.1155/2011/398918 (2010).
\end{thebibliography}

\end{document}
T